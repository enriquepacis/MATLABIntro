% ============================================
% ============================================
% ============================================
\section{Matrix Manipulations}

In MATLAB, all numbers are treated as matrices. The simplest matrix \texttt{A} is an $m \times n$ array of numbers, where $m$ refers to the number of (horizontal) rows and $n$ refers to the number of (vertical) columns in \texttt{A}. Even when we manipulate a single number, MATLAB treats this single number as a $1\times1$ matrix.

Matrices are important in solving systems of equations. They also are useful in MATLAB for storing arrays of numerical data. In this context, matrices may also be called ``storage arrays.''

% ============================================
% ============================================
\subsection{Defining Matrices and Vectors}
A matrix \texttt{A} can be formed by listing each element row by row within a single set of brackets (``\texttt{[}'' and ``\texttt{]}''). Within a row, elements may be separated by white space or by a column, and rows are delimited using semicolons. An example of this is given in the following command and output:
% vvv------------------------------------------------------------vvv
\begin{lstlisting}[style=Matlab-editor, label={MatrixDef01}, caption={A matrix is defined by enumerating its elements and delineating the matrix with brackets (\texttt{`[`} and \texttt{`]`}). Individual elements within a row are separated by a white space, or optionally a comma (\texttt{,}). Rows are separated using a semicolon (\texttt{;}).}]
>> A = [1 2 3 4; 5 6 7 8; 9 10 11 12] % this defines a 3x4 matrix A

A =

     1     2     3     4
     5     6     7     8
     9    10    11    12
\end{lstlisting}
% ^^^------------------------------------------------------------^^^

Alternately, functions such as \texttt{rand()}, \texttt{eye()}, \texttt{zeros()}, \texttt{ones()}, and \texttt{diag()} may be used to create matrices. Consider the following MATLAB script, entitled ``\texttt{MatrixCreation.m}'':
% vvv------------------------------------------------------------vvv
\begin{lstlisting}[style=Matlab-editor, label={MatrixDef02}, caption={Built-in MATLAB functions may be used to create/define matrices.}]
% MatrixCreation.m

m = 3; n = 2; B = rand(m) % defines a pseudorandom m-by-m matrix

C = rand([m,n]) % defines a pseudorandom m-by-n matrix

In = eye(n) % defines an n-by-n identity matrix

Znm = zeros(n, m) % defines an n-by-m matrix of zeros

Omn = ones(m, n) % defines a m-by-n matrix of ones

D = diag([1 2 3 4 5]) % defines a diagonal matrix with the input vector
                      %  on the diagonal
\end{lstlisting}
% ^^^------------------------------------------------------------^^^
Line 3 of \texttt{MatrixCreation.m} defines a $m\times m$ matrix of pseudorandom numbers using the \texttt{rand()} function and stores them in the variable \texttt{B}. Another way to use \texttt{rand()} is to specify a matrix input, where each element of the input matrix specifies a dimension size (the row specification is first, and the column specification comes second). This is done in line 5. In line 7, we use \texttt{eye(n)} to construct a $n\times n$ identity matrix. Line 9 creates a matrix of zeros, and line 11 creates a matrix of ones. Line 13 creates a diagonal matrix. Running \texttt{MatrixCreation.m} results in the following output:
% vvv------------------------------------------------------------vvv
\begin{lstlisting}[style=Matlab-editor, label={MatrixDef02out}, caption={The Command Window output for an invocation of Listing \ref{MatrixDef02}.}]
>> MatrixCreation

B =

    0.4898    0.7094    0.6797
    0.4456    0.7547    0.6551
    0.6463    0.2760    0.1626


C =

    0.1190    0.3404
    0.4984    0.5853
    0.9597    0.2238


In =

     1     0
     0     1


Znm =

     0     0     0
     0     0     0


Omn =

     1     1
     1     1
     1     1


D =

     1     0     0     0     0
     0     2     0     0     0
     0     0     3     0     0
     0     0     0     4     0
     0     0     0     0     5
\end{lstlisting}
% ^^^------------------------------------------------------------^^^

Row vectors are simply $m \times n$ matrices with $m=1$, and column vectors are $m\times n$ matrices with $n=1$. Here are some ways to construct numerical row vectors:
\begin{enumerate}
\item A simple row vector of $n$ integers may be formed by using the syntax ``\verb!k:k+n-1!''. For example, the input \verb!2:5! defines a vector \verb![2 3 4 5]!. This is referred to as a \textit{range} of integers, with unity increment between successive elements. A increment may be specified using ``\verb!min:inc:max!''. For example, the input ``\verb!2:2:10!'' specifies the row vector \verb![2 4 6 8 10]!
\item  A row  vector of floating-point double values may be formed using the syntax ``\verb!min:inc:max!'', where \texttt{min} is the low value, \texttt{max} is the lowest value and the first element in the vector, and each successive element larger than the previous value by \texttt{inc}. No values larger than \texttt{max} are included in the vector.
\item The \texttt{linspace()} command can be used to create row vector of $n$ evenly-spaced points that includes the minimum and maximum values \texttt{min} and \texttt{max} using the syntax ``\verb!linspace(min, max, n)!''
\end{enumerate}


% ============================================
% ============================================
\subsection{Subscripting Matrices}

MATLAB allows us to refer to individual elements of a matrix by using parentheses. If \texttt{A} is a matrix, then \verb!A(row_spec, col_spec)! provides a syntax for subscripting a subset of matrix elements. Here, \verb!row_spec! and \verb!col_spec! are row specifier and column specifier expressions, respectively. The \verb!row_spec! and \verb!col_spec! are also called \textit{indices} (plural of the word, ``index''), and subscripting is also called ``indexing.''

While \verb!row_spec! and \verb!col_spec! must be integers or logical values, there are many ways to specify rows and columns.
\begin{itemize}
\item If \verb!row_spec! and \verb!col_spec! both are single integers, \verb!A(row_spec, col_spec)! specifies a single element of \texttt{A}.
\item A subset of rows or columns may be specified using vectors or ranges. A full row-span or column-span may be achieved by using ``\texttt{:}'' as either the  \verb!row_spec! or \verb!col_spec! expression.
\item When using a vector as a \verb!row_spec! or \verb!col_spec! expression, the \texttt{end} keyword identifies the last valid index for the row or column.
\end{itemize}
Examples of matrix subscripting are given in the code listing of ``\texttt{MatrixSubscripting.m}'':
% vvv------------------------------------------------------------vvv
\begin{lstlisting}[style=Matlab-editor, label={MatrixSubscripting}, caption={Some examples of matrix subscripting, the referencing of a subset of matrix elements.}]
A = [1 2 3 4; 5 6 7 8; 9 10 11 12] % this defines a 3x4 matrix A

A34 = A(3,4) % extract the (3,4) element of A

B = A(2:3,1:3) % extract elements in rows 2-3 and columns 1-3 of A

v = A(:, 2) % extract elements in all rows and colum 2 of A [col. vector]

w = A(3, :) % extract elements row 3 and all columns of A [row vector]

A2end = A(2,end) % extract the last element of row 2

A2endm2_end = A(2, end-2:end) % extract the last three elements of row 2\end{lstlisting}
% ^^^------------------------------------------------------------^^^
When this script is run, the following output results:
% vvv------------------------------------------------------------vvv
\begin{lstlisting}[style=Matlab-editor, label={MatrixSubscriptingOut}, caption={Command Window output for Listing \ref{MatrixSubscripting}.}]
>> MatrixSubscripting

A =

     1     2     3     4
     5     6     7     8
     9    10    11    12


A34 =

    12


B =

     5     6     7
     9    10    11


v =

     2
     6
    10


w =

     9    10    11    12


A2end =

     8


A2endm2_end =

     6     7     8
\end{lstlisting}
% ^^^------------------------------------------------------------^^^

% ============================================
% ============================================
\subsection{Subscripting Matrices and Assignment Statements}

We can assign individual elements or subsets of elements of a matrix by using a subscripted matrix on the left-hand side (LHS) of an assignment statement. Consider the code listing of \texttt{SubscriptedAssignments.m}:
% vvv------------------------------------------------------------vvv
\begin{lstlisting}[style=Matlab-editor, label={MatrixSubscriptingAssign}, caption={Matrix subscripting may be used to assign values to a subset of matrix elements.}]
% SubscriptedAssignments.m

M = zeros(4, 5)

M(2, 5) = 1

M(1:2, 2:3) = 1

M(3:4, 2:4) = rand(2,2)
\end{lstlisting}
% ^^^------------------------------------------------------------^^^
Line 3 defines \texttt{M} as a $4\times 5$ matrix of zeros, with the following result:
% vvv------------------------------------------------------------vvv
\begin{lstlisting}[style=Matlab-editor]
M =

     0     0     0     0     0
     0     0     0     0     0
     0     0     0     0     0
     0     0     0     0     0
\end{lstlisting}
% ^^^------------------------------------------------------------^^^
 Line 5 of \texttt{SubscriptedAssignments.m} stores the value 1 in the $M_{2,5}$, the (2,5) element of \texttt{M}, with this output:
 % vvv------------------------------------------------------------vvv
\begin{lstlisting}[style=Matlab-editor]
M =

     0     0     0     0     0
     0     0     0     0     1
     0     0     0     0     0
     0     0     0     0     0
\end{lstlisting}
% ^^^------------------------------------------------------------^^^
A single scalar value can be assigned to multiple elements of a matrix \texttt{M} simultaneously. As an example of this, line 7 of \texttt{SubscriptedAssignments.m} assigns the single scalar value 1 to a $2\times2$ block of elements in \texttt{M}:
% vvv------------------------------------------------------------vvv
\begin{lstlisting}[style=Matlab-editor]
M =

     0     1     1     0     0
     0     1     1     0     1
     0     0     0     0     0
     0     0     0     0     0
\end{lstlisting}
% ^^^------------------------------------------------------------^^^
In general, however, the LHS and right-hand-side (RHS) of an assignment statement must match in size. Line 9 of \texttt{SubscriptedAssignments.m} demonstrates this: MATLAB throws an error because the target of the assignment on the LHS of line 9 is a $2\times3$ block of elements, but the value on the RHS of line 9 is a $2\times2$ block.
% vvv------------------------------------------------------------vvv
\begin{lstlisting}[style=Matlab-editor]
Subscripted assignment dimension mismatch.

Error in SubscriptedAssignments (line 9)
M(3:4, 2:4) = rand(2,2)
\end{lstlisting}
% ^^^------------------------------------------------------------^^^

% ============================================
% ============================================
\subsection{Concatenation}
Matrices can be combined via concatenation. Horizontal concatenation is the act of creating a larger matrix \texttt{C} from two smaller matrices \texttt{A} and \texttt{B} by setting \texttt{A} and \texttt{B} side-by-side. The syntax for horizontal concatenation is \verb!{C = [A B]!, or equivalently, \verb!{C = [A, B]!. This involves listing the constituent matrices within brackets (\texttt{[} and \texttt{]}) and separating them with either a comma (\texttt{,}) or a white space. An example of this is:
% vvv------------------------------------------------------------vvv
\begin{lstlisting}[style=Matlab-editor, label={MatrixHorizCat}, caption={Matrices may be concatenated horizontally by using brackets and either a comma or a white space to delimit individual elements.}]
>> A = [1 2 3]; B = [4 5 6]; C = [A B]

C =

     1     2     3     4     5     6
\end{lstlisting}
% ^^^------------------------------------------------------------^^^
In the input command line, we define \texttt{A} and \texttt{B} as row vectors, which themselves are horizontal concatenations of three numbers, each. Then, we form \texttt{C} by horizontally concatenating \texttt{A} and \texttt{B}. Horizontal concatenation requires that constituents have the same number of rows. Concatenation can be thought of as making a matrix of matrices.

Vertical concatenation is accomplished in the same was as horizontal concatenation, except that constituent matrices must have the same number of columns, and constituent matrices must be separated by a semicolon ('\texttt{;}'). The following listing forms the matrix \texttt{D} by vertically concatenates \texttt{A} and \texttt{B}:
% vvv------------------------------------------------------------vvv
\begin{lstlisting}[style=Matlab-editor, label={MatrixVertCat}, caption={Matrices may be concatenated vertically by using brackets and either a semicolon to delimit individual elements.}]
>> D = [A; B]

D =

     1     2     3
     4     5     6
\end{lstlisting}
% ^^^------------------------------------------------------------^^^

 
% ============================================
% ============================================
\subsection{Single-subscripting}
Matrices also may be subscripted with a single index, instead of using both \texttt{row\_spec} and \texttt{col\_spec} specifiers. For an $m\times n$ matrix \texttt{A}, the element \texttt{A(1)} is the $A_{1,1}$ element in the upper left. MATLAB is known as a \textit{column-major} language, in which elements are counted down the columns. Thus, \texttt{A(2)} refers to $A_{2,1}$, which is the element immediately below $A_{1,1}$ in column 1; and \texttt{A(3)} refers to $A_{2,1}$. When the end of the column is reached, the counting continues at the top of the next column.

A simple application of this is the subscripting of a vector \texttt{A}. The expression \texttt{A(k)} refers to the $k$-th element of \texttt{A}, regardless of whether \texttt{A} is a row vector or a column vector. 
 % ============================================
% ============================================
\exmp{Example} \label{Exmp:Cosines}
Define a row vector \texttt{t} that includes $n_t = 75$ time points from $t=0$ to $t=1$ (inclusive). Then, define a matrix \texttt{fn} to represent $f_n (t) = \cos (2\pi n t)$ over the interval $t\in [0, 1]$ for $n \in \{1, 2, 3, 4\}$.

\noindent \underline{Solution}
The solution is given in the code listing of \texttt{Sinusoids.m}.
% vvv------------------------------------------------------------vvv
\begin{lstlisting}[style=Matlab-editor]
% Sinusoids.m
%
% By E.P. Blair
% Baylor University
%

nt = 75; % number of time points
t = linspace(0, 1, nt); % [s] define a time vector

fn = zeros(4, nt); % Define a storage array to hold fn calculations
                   % The n^th row of fn stores fn = cos(2*pi*n*t)
      
% calculate fn and store fn for each value of n in the n^th row
fn(1,:) = cos(2*pi*t); % n = 1
fn(2,:) = cos(2*pi*2*t); % n = 2
fn(3,:) = cos(2*pi*3*t); % n = 3
fn(4,:) = cos(2*pi*4*t); % n = 4
\end{lstlisting}
% ^^^------------------------------------------------------------^^^

So far, we have discussed MATLAB numerical data. Numbers in MATLAB are double-precision floating-point values, or simply \verb!double!. MATLAB uses 8 bytes of memory to store each one. When storing complex numbers, MATLAB uses two doubles for each individual number: one for the real part, and one for the complex part. 

% ============================================
% ============================================
\subsection{The \texttt{char} Data Type and Strings}

Now we consider data of type \verb!char!, which is short for ``character.'' MATLAB \texttt{char}s store individual letters (uppercase [\texttt{A}-\texttt{Z}] or lowercase [\texttt{a}-\texttt{z}]), numerals (\texttt{0}-\texttt{9}), and special characters (\verb!~!, \verb!!!, \verb!@!,  \verb!#!, etc.). Data of type \verb!char! may be stored in a variable using an assignment statement. In MATLAB, we identify a character using a pair of single quotes, as follows in this MATLAB command line input and resulting output:
% vvv------------------------------------------------------------vvv
\begin{lstlisting}[style=Matlab-editor, caption={Simple definition of \texttt{char} variables in the Command Window.}]
>> x = '1'

x =

    '1'

>> y = 1

y =

     1

>> 
\end{lstlisting}
% ^^^------------------------------------------------------------^^^
Here, we have stored the \texttt{char} data \texttt{'1'} in the variable \texttt{x} (line 1, with output on line 3). We contrast this with the case where we store numerical (\texttt{double} data (the value 1) in the variable \texttt{y} (line 7, with output on line 9). The \texttt{char} data \verb!'1'! is distinguished from the \texttt{double} data 1 by the single quotes.


\subsubsection{Strings}
A string is simply an array of \texttt{char}s. A string may be defined using the following syntax:
% vvv------------------------------------------------------------vvv
\begin{lstlisting}[style=Matlab-editor, caption={Simple definition of a \texttt{char} string and its output.}]
>> x = 'hello world'

x =

    'hello world'
\end{lstlisting}
% ^^^------------------------------------------------------------^^^
It is most typical to work with strings as row vectors, but column vectors and general matrices also may be constructed. Matrix concatenations are available for strings:
% vvv------------------------------------------------------------vvv
\begin{lstlisting}[style=Matlab-editor, caption={Horizontal and vertical string concatenations.}]
>> x = 'cat'; y = 'dog'; z = [x y]

z =

    'catdog'
>> w = [x; y]

w =

  2�3 char array

    'cat'
    'dog'
\end{lstlisting}
% ^^^------------------------------------------------------------^^^

